\chapter{Ukážková kapitola}

Každá kapitola má nejaký abstrakt.

Po abstrakte nasleduje intro.

\section{Prvá sekcia ukážkovej kapitoly}

Následne je obsah členený do sekcíí.

\section{Druhá sekcia s podsekciou}

Niekedy má sekcia široký obsah, a preto je vhodné využiť aj nejaké podsekcie.

\subsection{Podsekcia druhej sekcie}
Toto vytvára tretiu úroveň.

\subsection{Podsekcie majú limity}
Môžeš mať viacero podsekcí, ale nižšie už neodporúčam ísť. Tretia úroveň je konečná.

\section{Druhá sekcia}

\begin{itemize}
    \item \href{http://www.www2015.it/documents/proceedings/proceedings/p1373.pdf}{Daily-Aware Personalized Recommendation based on Feature-Level Time Series Analysis}
    \item Sevcech DizP - hladanie vzoru v prude dat a ich kompresia
\end{itemize}

Na konci by mala byť bibliografia.

\renewcommand{\myBibliography}[0] {./references.bib}
\printbibliography[heading=references,segment=\therefsegment,resetnumbers=true]